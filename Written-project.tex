\documentclass[]{article}
\usepackage{lmodern}
\usepackage{amssymb,amsmath}
\usepackage{ifxetex,ifluatex}
\usepackage{fixltx2e} % provides \textsubscript
\ifnum 0\ifxetex 1\fi\ifluatex 1\fi=0 % if pdftex
  \usepackage[T1]{fontenc}
  \usepackage[utf8]{inputenc}
\else % if luatex or xelatex
  \ifxetex
    \usepackage{mathspec}
  \else
    \usepackage{fontspec}
  \fi
  \defaultfontfeatures{Ligatures=TeX,Scale=MatchLowercase}
\fi
% use upquote if available, for straight quotes in verbatim environments
\IfFileExists{upquote.sty}{\usepackage{upquote}}{}
% use microtype if available
\IfFileExists{microtype.sty}{%
\usepackage{microtype}
\UseMicrotypeSet[protrusion]{basicmath} % disable protrusion for tt fonts
}{}
\usepackage[margin=1in]{geometry}
\usepackage{hyperref}
\hypersetup{unicode=true,
            pdftitle={Expected Goal Analysis},
            pdfauthor={Oriol Garrobé and Dávid Hrabovszki},
            pdfborder={0 0 0},
            breaklinks=true}
\urlstyle{same}  % don't use monospace font for urls
\usepackage{graphicx,grffile}
\makeatletter
\def\maxwidth{\ifdim\Gin@nat@width>\linewidth\linewidth\else\Gin@nat@width\fi}
\def\maxheight{\ifdim\Gin@nat@height>\textheight\textheight\else\Gin@nat@height\fi}
\makeatother
% Scale images if necessary, so that they will not overflow the page
% margins by default, and it is still possible to overwrite the defaults
% using explicit options in \includegraphics[width, height, ...]{}
\setkeys{Gin}{width=\maxwidth,height=\maxheight,keepaspectratio}
\IfFileExists{parskip.sty}{%
\usepackage{parskip}
}{% else
\setlength{\parindent}{0pt}
\setlength{\parskip}{6pt plus 2pt minus 1pt}
}
\setlength{\emergencystretch}{3em}  % prevent overfull lines
\providecommand{\tightlist}{%
  \setlength{\itemsep}{0pt}\setlength{\parskip}{0pt}}
\setcounter{secnumdepth}{0}
% Redefines (sub)paragraphs to behave more like sections
\ifx\paragraph\undefined\else
\let\oldparagraph\paragraph
\renewcommand{\paragraph}[1]{\oldparagraph{#1}\mbox{}}
\fi
\ifx\subparagraph\undefined\else
\let\oldsubparagraph\subparagraph
\renewcommand{\subparagraph}[1]{\oldsubparagraph{#1}\mbox{}}
\fi

%%% Use protect on footnotes to avoid problems with footnotes in titles
\let\rmarkdownfootnote\footnote%
\def\footnote{\protect\rmarkdownfootnote}

%%% Change title format to be more compact
\usepackage{titling}

% Create subtitle command for use in maketitle
\providecommand{\subtitle}[1]{
  \posttitle{
    \begin{center}\large#1\end{center}
    }
}

\setlength{\droptitle}{-2em}

  \title{Expected Goal Analysis}
    \pretitle{\vspace{\droptitle}\centering\huge}
  \posttitle{\par}
    \author{Oriol Garrobé and Dávid Hrabovszki}
    \preauthor{\centering\large\emph}
  \postauthor{\par}
      \predate{\centering\large\emph}
  \postdate{\par}
    \date{28/08/2020}

\usepackage{booktabs}
\usepackage{longtable}
\usepackage{array}
\usepackage{multirow}
\usepackage{wrapfig}
\usepackage{float}
\usepackage{colortbl}
\usepackage{pdflscape}
\usepackage{tabu}
\usepackage{threeparttable}
\usepackage{threeparttablex}
\usepackage[normalem]{ulem}
\usepackage{makecell}
\usepackage{xcolor}

\begin{document}
\maketitle

\hypertarget{abstract}{%
\section{Abstract}\label{abstract}}

The world is changing with the advance of technology and the vast amount
of data available, and so does sport. Team sports such as football, are
starting to introduce mathematical methods to analize performances,
recognize trends and patterns and predict results. From this point, the
purpose of this study is to develop a Machine Learning model that is
capable to predict when a shot is going to end up being a goal.

To do so, many different mathematical approaches are trained on a
dataset that tries to represent the exact moment of the shot as detailed
as possible. Also, the dataset includes information about the players,
their skills and the moment of the game. By doing so, a Logistic
Regression algorithm is the one that yields the best performance when
predicting goals, with an F1-score of 35.24\%, improving the previous
works on the field.

This study intends to provide new information and hidden insights to
professionals - players, coaches or managers - in order to improve the
game, looking for a better decision process when it comes to score a
goal.

\hypertarget{introduction}{%
\section{Introduction}\label{introduction}}

Trying to play the best football possible in order to win all the titles
is a quest that all the best teams in the world follow. Practicing
methods have changed, and therefore football istelf. Because of the
evolution of technology and the rapidly increasing amount of data - as
done in other sports such as baseball - football is introducing
datascience techniques to its toobox in order to improve the game.

Teams use data science and machine learning approaches to analyse which
aspects of the games can be improved to become the best team. Artificial
intelligence (AI) in sports is very relevant nowadays, and it changed
the way coaches approach their practice sessions.

From this point of view, a large number of researchers have worked on
sports analytics and in particular in football trying to build the state
of the art models to find the hidden insights from the sport that the
human eye cannot see by analysing data.

This project aims to study the probability of scoring of a player when
shooting, so professionals can use this approach to look for better
positions of shooting - those situations when a player has a big chance
of scoring - and avoid those that apparently look good but they have a
lower chance of goal.

As mentioned, there is a bast number of papers regarding football, but
there is still room fore more. There are so many aspects of the game
that can be analysed, and the game can be heavily improved using AI.
From this regards, this project is based in a metric from StatsBomb
called expected Goal (Larrousse 2019). The aim is to work over this
metric and try to improve it by adding some other variables and try
different machine learning models that could give a better result.

The data used is the datased provided by StatsBomb itself {[}reference
this shit{]}. Apart from the StatsBomb dataset, aiming to go one step
further, ratings from the players in the game are used, since it is
believed that are players that in the same circumstances achieve
different results. This means that not only the scenario is considered
in this project but also the actors. For this purpose player ratings are
included to the dataset coming from the FIFA video game.

A more sophisticated approach to analyse the shot situations in a
football game would clearly help practitioners to design more technical
interventions and strength and conditioning programmes for players.
Accordingly, it is the purpose of this study to develop and validate a
machine learning algorithm to identify the best shooting situations for
players.

\hypertarget{related-work}{%
\section{Related work}\label{related-work}}

\hypertarget{methods}{%
\section{Methods}\label{methods}}

\hypertarget{dataset}{%
\subsection{Dataset}\label{dataset}}

The project is mainly based on a dataset from StatsBomb Academy
(Academy, n.d.). This is a free dataset provided in order to new
research projects in football analytics. The data from statsBomb
includes very detailed and interesting features relevant for the project
such as: location of the players on the pitch in any shot - including he
position and actions of the Goalkeeper-, detailed information on
defensive players applying pressure on the player in possession, or
which foot the player on possession uses among others.

The data is provided as JSON files exported from the StatsBomb Data API,
in the following structure:

\begin{itemize}
\item
  Competition and seasons stored in competitions.json.
\item
  Matches for each competition and season, stored in matches. Each
  folder within is named for a competition ID, each file is named for a
  season ID within that competition.
\item
  Events for each match, stored in events. The file is named for a match
  ID.
\item
  Lineups for each match, stored in lineups. The file is named for a
  match ID.
\end{itemize}

In particular, the dataset only contains information about F.C.
Barcelona games in the spanish national championship La Liga from 2005
up until 2019.

As stated earlier, player skills are also considered in this project. To
have an unbiased rate of players, ratings from the most played around
the world football video game FIFA from EA Sports (Arts, n.d.) are used.
Also, it is considered which is the preferred foot of the player,
information sourced from the same database. More in particular, this
players information is gotten from \emph{FIFAindex} (FIFAindex, n.d.).

\hypertarget{data-preprocessing}{%
\subsubsection{Data Preprocessing}\label{data-preprocessing}}

Data from StatsBombs comes in JSON format. One easy way to work with
JSON data in R is through the \textbf{jsonlite} package that transforms
JSON data - which is a combination of nested lists - to nested
dataframes.

The StatsBomb dataset containing all the relevant information for the
project is the \textbf{Events} dataset. Each data point in this data set
is an event that occurred in a specific game, such as: pass, block,
dribble or shots, among others.

The first step working with the dataset is to erase all the incomplete
datapoints or those with wrong information that would afterwards make
the models fail. From this point, and because this project is only
focused on shots, the dataset is filtered by the variable \textbf{type}
which contains the type of action of the event. This variable is
filtered so only the events regarding shots remain. A \textbf{shots}
dataset, way lighter than the one used before is created.

It is important to emphasize on the fact that this dataset contains an
extremely useful information which is enclosed in the variable called
\textbf{shot.freeze\_frame}. This \textbf{shot.freeze\_frame} states
where all the players are positioned on the pitch at the moment of the
event. It is considered that relevant since it allows to make a perfect
picture of the scenario at the moment of the shot.

From this regards, and using the information in
\textbf{shot.freeze\_frame}, a number of new features can be computed
and added to the dataset. More in particular, geometric features
regarding the position of the striker, the defenders and the goalkeeper
are computed. Such features will allow to know, for instance, which is
the distance to target of the ball, whether the goalkeeper is properly
positioned or if there are defenders close enough to disturb the
striker.

Finally, every data point from the \emph{shots} dataset, containing also
the geometric features computed, is complemented with information of the
players. For instance, the ratings of the striker and the goalkeeper in
action are added or whether the player shooting is using his preferred
foot or not.

Finally, some features from the \textbf{shots} dataset which are not
relevant to study are removed, creating a dataset for analysis called
\textbf{data} that will be the one used to train the models. The final
features in the \textbf{data} dataset are:

\begin{verbatim}
## 
## Attaching package: 'kableExtra'
\end{verbatim}

\begin{verbatim}
## The following object is masked from 'package:dplyr':
## 
##     group_rows
\end{verbatim}

\begin{table}

\caption{\label{tab:unnamed-chunk-1}Variables from the dataset defined.}
\centering
\begin{tabular}[t]{>{\bfseries\raggedright\arraybackslash}p{3cm}|l}
\hline
Variable & Definition\\
\hline
id & Numeric. Number representing a unique player.\\
\hline
strong\_foot & Boolean. States whether the player use the strong foot or not.\\
\hline
overall\_rating & Numeric. Overall rating from FIFA ratings.\\
\hline
shot.power & Numeric. Shot power from FIFA ratings.\\
\hline
shot.finishing & Numeric. Shot finishing from FIFA ratings.\\
\hline
gk\_rating & Numeric. Overall rating for goalkeepers from FIFA ratings.\\
\hline
gk.reflexes & Numeric. Goalkeeper reflexes from FIFA ratings.\\
\hline
gk.rushing & Numeric. Goalkeeper rushing from FIFA ratings.\\
\hline
gk.handling & Numeric. Goalkeeper handling from FIFA ratings.\\
\hline
gk.positioning & Numeric. Goalkeeper positioning from FIFA ratings.\\
\hline
shot.first\_time & Boolean. States if the shot is the first of the game for the given player.\\
\hline
under\_pressure & Boolean. States if the player shooting has opponents close enough to disturb the shot.\\
\hline
home & Boolean. States whether the player shooting is playing home or away.\\
\hline
dist & Numeric. distance to center of the goal from shot taker.\\
\hline
angle & Numeric. angle of the goal from the from shot taker (in degrees).\\
\hline
obstacles & Numeric. number of players (teammates \& opponents NOT including GK) between goal and shot taker (inside the triangle of the goalposts and the shot taker).\\
\hline
pressure\_prox & Numeric. Distance from closest opponent to shooter.\\
\hline
pressure\_block & Boolean. Can the goalkeeper save the shot by being inside the triangle.\\
\hline
gk\_obstacle & Boolean. States whether the goalkeeper is between the ball and the goal.\\
\hline
gk\_pos & Numeric. goalkeeper's positioning, best if gk is standing on the line that halves the angle of the shot (value between 0 (angle is the same), to 1 (angle is halved).\\
\hline
gk\_pos\_adjusted & Numeric. same as gk\_pos, but it is less strict with shots with a tight angle.\\
\hline
gk\_dist\_from\_player & Numeric. distance between goalkeeper and shot taker.\\
\hline
gk\_dist\_from\_goal & Numeric. distance between goalkeeper and the center of the goal.\\
\hline
goal & Binary. 1 the shot is goal, 0 the shot is not goal.\\
\hline
\end{tabular}
\end{table}

\hypertarget{feature-engineering}{%
\subsubsection{Feature engineering}\label{feature-engineering}}

We created new features based on player locations for every shot in the
dataset, similar to traditional xG metrics. The difference is that we
calculated many more variables that the traditional approaches do not
take into account. The statsbomb dataset contains the x and y
coordinates of the shooter and other players that are relevant to the
shot (both teammates and opponents including the goalkeeper). In the
next part we explain what features we created and how.

\hypertarget{distance}{%
\paragraph{Distance}\label{distance}}

Euclidean distance between the shooter and the center of the goal.

\hypertarget{angle}{%
\paragraph{Angle}\label{angle}}

Angle of the goal from the shooter's point of view in degrees. Picturing
a triangle, where one side is the goal line and the other two are the
imaginary lines connecting each goalpost to the shooter, we need the
angle opposite of the goal line. We used the following formula:

\[
\alpha = \arccos\bigg(\frac{b^2+c^2-a^2}{2\cdot b\cdot c}\bigg)\cdot\frac{180}{\pi}
\]

where a is the goal line, and b and c are the imaginary lines between
the shooter and the posts (Calculator, n.d.).

\hypertarget{obstacles}{%
\paragraph{Obstacles}\label{obstacles}}

Number of players (teammates and opponents not including the opponent
goalkeeper) between the goal and shooter -- in other words, inside the
triangle defined by the shooter and the goalposts. To calculate this, we
first need to evaluate if a player is inside said triangle or not. If
the area of this triangle is equal to the sum of the partial triangles
defined by:

\begin{enumerate}
\def\labelenumi{\arabic{enumi}.}
\item
  The shooter, one goalpost and the player being evaluated
\item
  The shooter, the other goalpost and the player being evaluated
\item
  The two goalposts and the player being evaluated
\end{enumerate}

Then we conclude that the player being evaluated is inside the main
triangle, therefore he is an obstacle. To calculate the area of a
triangle based on the coordinates of its points, we used the following
formula in R:

\[
area = \left\lvert \frac{a_1\cdot(b_2-c_2)+b_1\cdot(c_2-a_2)+c_1\cdot(a_2-b_2)}{2} \right\rvert 
\]

where a, b and c are numeric vectors of length 2 representing the points
of the triangle (Geeks, n.d.).

\hypertarget{pressure-proximity}{%
\paragraph{Pressure proximity}\label{pressure-proximity}}

The Euclidean distance between the shooter and the opponent closest to
him.

\hypertarget{pressure-block}{%
\paragraph{Pressure block}\label{pressure-block}}

The closest opponent's physical ability to block the shot by being
inside the triangle defined by the shooter and the goalposts. Boolean
value.

\hypertarget{goalkeeper-obstacle}{%
\paragraph{Goalkeeper obstacle}\label{goalkeeper-obstacle}}

The opponent goalkeeper's physical ability to save the shot by being
inside the triangle defined by the shooter and the goalposts. Boolean
value.

\hypertarget{goalkeeper-positioning}{%
\paragraph{Goalkeeper positioning}\label{goalkeeper-positioning}}

Positioning of the opponent goalkeeper. The value is between 0 and 1,
where a value of 1 means that the goalkeeper halves the angle of the
shot, while a value of 0 means that the angle remains the same. This
does not take into account if the goalkeeper is standing on the line or
right in front of the shooter, only that he is positioned on the line
that halves the angle of the shot. A larger kernel width means that the
kernel distinguishes less between bad and good goalkeeper positioning,
while a small value means that only split angles close to 0.5 can get a
good mark for goalkeeper positioning, as it can be observed on Figure
\ref{fig:gauss_kernel}. We used a kernel width of 0.2 for this feature.

\begin{figure}[!h]

{\centering \includegraphics[width=1\linewidth]{Written-project_files/figure-latex/unnamed-chunk-4-1} 

}

\caption{\label{fig:gauss_kernel} Gaussian kernels that give more weight to values around 0.5 (good split angle by the goalkeeper). Kernel width controls how strict the kernel is.}\label{fig:unnamed-chunk-4}
\end{figure}

This feature was not included in the analysis, because it proved to be
too strict when dealing with tighter shots.

\hypertarget{goalkeeper-positioning-adjusted}{%
\paragraph{Goalkeeper positioning
adjusted}\label{goalkeeper-positioning-adjusted}}

Very similar to Goalkeeper positioning, but this feature is adjusted
with the shot angle, meaning that it takes into account the full shot
angle when evaluating goalkeeper positioning. This is necessary, because
the previous variable was too strict when it came to tight shots, and it
gave a bad value for the goalkeeper, even though he was still positioned
pretty well. For example, if the angle of the shot was 5 degrees, the
goalkeeper should not be given a bad mark for his positioning if he
splits the angle to 1 and 4 degrees. With such tight angles, it does not
really matter where he stands, as long as he is obstructing the goal, of
course.

We solved this problem by introducing another Gaussian kernel that gives
a high output value for small input values (shot angles) and outputs a
value close to 0.2 for larger input values (shot angles). The kernel
width in this case was chosen to be 20. Then, this kernel value was used
as the kernel width for the original Gaussian kernel described above in
the previous feature. This way we achieved that the Goalkeeper
positioning adjusted feature is more lenient towards tight shots than
Goalkeeper positioning, thus more accurate in evaluating real life
goalkeeper positioning.

\begin{figure}[!h]

{\centering \includegraphics[width=0.65\linewidth]{Written-project_files/figure-latex/unnamed-chunk-5-1} 

}

\caption{\label{fig:gauss_kernel_adjust} Adjusting Gaussian kernel that gives more weight to lower values (shot angles)}\label{fig:unnamed-chunk-5}
\end{figure}

\hypertarget{goalkeeper-distance-from-player}{%
\paragraph{Goalkeeper distance from
player}\label{goalkeeper-distance-from-player}}

Euclidean distance between shooter and goalkeeper.

\hypertarget{goalkeeper-distance-from-goal}{%
\paragraph{Goalkeeper distance from
goal}\label{goalkeeper-distance-from-goal}}

Euclidean distance between center of the goal and goalkeeper.

\hypertarget{machine-learning-models}{%
\subsection{Machine Learning Models}\label{machine-learning-models}}

\hypertarget{logistic-regression}{%
\subsubsection{Logistic Regression}\label{logistic-regression}}

From the moment that one tries to develop a classification algortihm,
the first choice usually is the Logistic Regression. This study is no
different.

The Logistic Regression model it is based on the logistic function
(sigmoid) which is an S-shaped curve that takes any real number and maps
it between 0 and 1, but never reaching this numbers. From this point, by
setting a threshold, also between 0 and 1, it is possible to divide data
points in two classes. For instance, if the threshold is set at 0.7,
those data points with a probability higher than 0.7 will be classified
as one class and those with a probability lower than 0.7 will be
classified as the other. The regression coefficients (betas) of the
Logistic Regression are estimated by training the data.

In particular, the threshold set for this study a threshold of 0.5. In
order to run the Logistic Regression algorithm, the package \textbf{glm}
(Schlegel 2019) from CRAN is used.

\hypertarget{random-forest-and-adaboost}{%
\subsubsection{Random Forest and
AdaBoost}\label{random-forest-and-adaboost}}

Both Random Forest and AdaBoost models are very good choices when
training a classification model. This models are built on the decision
tree model.

The Random Forest model in particular generates a number of trees based
on a bootstrapped subset of the sample and only considering a subset of
variables at each step. The result is a wide variety of trees. From this
point the data is run over all trees and that yield a decision, the
decision given by more trees is the one that prevails. This is called
Bagging. To run the Random Forest model, the package
\textbf{randomForest} (Breiman and Wiener 2018) from CRAN is used.

AdaBoost on the other side also generates a number of trees, but only
trees that have a root node and two leaves with no children, this trees
are called stumps and are not great at making accurate classification,
they are weak learners. From this point, once one stump is made the data
is run through it, and a decision is made. The errors made by this stump
influence when creating the next one and so on. Therefore, some stumps
have a greater impact to the model. To run the AdaBoost model, the
package \textbf{mboost} (Hofner 2020) from CRAN is used.

In order to get the best model possible and not to make it too
computational expensive, in both models it is important to choose the
optimal number of trees. To do so, both algorithms are run a number of
times with different amount of trees. Based on the error rates, the best
model in is chosen, and afterwords trained with the data to get the best
results possible. In figure \ref{fig:trees} it can be seen the error
rates against the number of trees. The smallest error rate for training
data in the Random Forest model appears when 40 trees are created, and
for the AdaBoost model it appears when 30 trees are used. Therefore, the
final models are set.

\begin{figure}[!h]

{\centering \includegraphics[width=0.65\linewidth]{Written-project_files/figure-latex/unnamed-chunk-6-1} 

}

\caption{\label{fig:trees} Random Forest and AdaBoost models error rates against the number of trees used.}\label{fig:unnamed-chunk-6}
\end{figure}

\hypertarget{knn-model}{%
\subsubsection{KNN model}\label{knn-model}}

The KNN (K-Nearest Neighbours) model is a classification model that
identifies the number k of closest labelled points to the one studying
and estimates its class. The class chosen is the one with the bigger
amount of neighbors with said class. This is a powerful algorithm but it
is very important to choose the optimal number of neighbours, in other
words, choose the value of k. This value is chosen with Cross Validation
(CV), which is a model validation which partitions the data in
complementary subsets, performs analysis on one subsets and validates on
the others in order to give a more accurate model. The final number of
neighbours used in the KNN algorithm is 9, which gives the best result
for prediction. To run the KNN algorithm with CV, the package
\textbf{kknn} from CRAN is used.

\hypertarget{results}{%
\section{Results}\label{results}}

\begin{itemize}
\item
  \textbf{Accuracy}. Stands for the ratio of correctly predicted
  observation over the total observations. If it is high it means that
  there are a lot of good predictions but it must be taken into account
  that for assymetric datasets or uneven classes - those where one class
  is larger than the other - it can be biased. In particular, in the
  dataset used there are way less shots that ended up being a goal than
  those that not. So, with a naive predictor all the datapoints could
  have been set to not goal/miss and the accuracy would have been still
  high. That is other metrics must be used in order to take this into
  account.
\item
  \textbf{Recall}. Otherwise called sensitivity, stands for the
  correctly predicted positive observations to all observations in an
  actual class. This metric checks how many properly predicted points
  are predicted within a class, in this case \textbf{goal}. This means
  that it tries to show how many predicted goals are actually goals.
  Since in football there are not many goals, it is a very innacurate
  metric, as many times a situation that should clearly end up in goal
  in the end is not. This is why this one is not a very indicative
  metric for the purposes or the project. However, it is good to compute
  it in order to get some conclusions.
\item
  \textbf{F1-score}. This metric is a weighted average of precision and
  recall, being precision the number of properly predicted observations
  in a class over all the observations fom that class. From this point,
  it takes both false positives and false negatives into account too, it
  is very useful when the goal is to predict uneven classes. In this
  case, as there are way more observations that are not goal than those
  that are goal, this metric is the one that mirrors best the quality of
  the algorithm.
\end{itemize}

\hypertarget{discusion}{%
\section{Discusion}\label{discusion}}

In this study, a Machine Learning algorithm to quantify the probability
of a football player scoring is develooped, using the positioning of the
players in the pitch and their skills. 4 classification approaches using
25 variables which derived from the particular scenario at the moment of
the shot were examined. It has been proven that it is possible to
classify whether a shot is going to be a goal or not with good accuracy,
being the best performing method the logistic regression with an F-1
score of 35.24\%, a recall of 23.72\% and an accuracy of 85.89\%. It
also has improved the xG metric from statsBomb by adding player skills
to the dataset and testing other Machine Learning approaches. This will
provide professional of the sport such as players, coaches or managers
some insights from the game that can be very useful to improve their
performance.

Models based on decision trees such as Random Forest or AdaBoost
provided mixed results that did not improve the previous works on the
field. In particular, the optimal Random Forest algorithm with 40 trees
yielded an F1-score of 32.24\%, a recall of 22.05\% and an accuracy of
60\%. Looking at this results, it can be seen that eventhough the
F1-Score is close to the one from xG, the overall accuracy is poorer,
which means that many shots that were classified as miss where actually
a goal. On the other hand, the optimal AdaBoost algorithm with 30 trees
yielded an F1-score of 19.71\%, a recall of 11.18\% and an accuracy of
83.33\%. All the results are worst than the xG metric, bringing no
improvement to the field. This model failed as almost no goals were
predicted, only 11.18\% (recall) of the goals were predicted properly.
From this point of view, these two algorithms do not bring improvements
to the field or relevant information that can be used to improve the
game.

Another model used to classfy goals is the K-NN algorithm. This
algorithm yielded better results than models based on decision trees
with an F1-score of 39.54\%, a recall of 27.79\% and an accuracy of
68.50\%. It has the best F1-Score of all models used and also improves
the xG metric. As stated before, the F1-score is the metric that mirrors
best the quality of this algorithm, and this method could be the one
chosen for the project. It also has the best recall among all the
models. However, it has a relevant lower overall accuracy compared to xG
or Logistic Regression. It yields a better F1-score and a better recall
due the fact that predicts many more situations as goals than Random
Forest, Adaboost or xG. However, many situations that were not goal were
classified as goal, this is why this model has a lot of room to improve.

Finally, \#\#\# HERE WRITE ABOUT WHY WE CHOOSE LOGISTIC REGRESSION
Notes: could have lowered threshold below 50\% to get more predicted
goals, but it doesn't make sense, and lowers accuracy (check if it
lowers F1 score)

\hypertarget{player-performances}{%
\subsection{Player performances}\label{player-performances}}

In the previous sections we created models to predict goals, chose the
best one, and now we are going to interpret the results on the player's
level. We aim to find out how well each player performs from the aspect
of converting shots to goals. We suspect that there are players that
score more goals than they are expected to, and there are some that
score fewer. It is important to note, that the database we used for
analysis had 1044 goals, while we only predicted 320 goals with our best
model for the whole dataset. Therefore, to make the amount of actual and
predicted goals by each player comparable, we scaled up the amount of
predicted goals for the sake of this performance analysis, so they also
sum up to 1044.

The results can be observed in the table below, which is ordered by the
amount of goals scored. It is no surprise that Messi scored 33.85 more
goals (11\% more) than he should have, based on his chances. This is in
line with our belief that he is a very efficient striker. Even more
efficient than Messi are Ivan Rakitic and Daniel Alves. The latter
player is primarily a defender, but still managed to score 11 goals,
even though none were predicted for him.

The most underperforming players, when it comes to converting chances,
were Samuel Eto'o and Zlatan Ibrahimovic. Luis Suárez also scored much
fewer goals (27.76), than expected.

\begin{table}

\caption{\label{tab:unnamed-chunk-8}Player performances (who scored more than 10 goals)}
\centering
\begin{tabular}[t]{l|r|r|r|r}
\hline
Name & Goals predicted (scaled up) & Goals scored & Diff. (amount) & Diff. (ratio)\\
\hline
Lionel Andrés Messi Cuccittini & 300.15 & 334 & 33.85 & 0.11\\
\hline
Luis Alberto Suárez Díaz & 133.76 & 106 & -27.76 & -0.21\\
\hline
Samuel Eto"o Fils & 104.40 & 62 & -42.40 & -0.41\\
\hline
Pedro Eliezer Rodríguez Ledesma & 55.46 & 48 & -7.46 & -0.13\\
\hline
Neymar da Silva Santos Junior & 58.73 & 46 & -12.73 & -0.22\\
\hline
Thierry Henry & 42.41 & 32 & -10.41 & -0.25\\
\hline
David Villa Sánchez & 35.89 & 31 & -4.89 & -0.14\\
\hline
Xavier Hernández Creus & 26.10 & 31 & 4.90 & 0.19\\
\hline
Alexis Alejandro Sánchez Sánchez & 19.58 & 29 & 9.42 & 0.48\\
\hline
Andrés Iniesta Luján & 22.84 & 25 & 2.16 & 0.09\\
\hline
Gerard Piqué Bernabéu & 13.05 & 25 & 11.95 & 0.92\\
\hline
Ivan Rakitić & 6.53 & 22 & 15.47 & 2.37\\
\hline
Francesc Fàbregas i Soler & 13.05 & 20 & 6.95 & 0.53\\
\hline
Bojan Krkíc Pérez & 9.79 & 19 & 9.21 & 0.94\\
\hline
Seydou Kéita & 9.79 & 12 & 2.21 & 0.23\\
\hline
Zlatan Ibrahimović & 19.58 & 12 & -7.58 & -0.39\\
\hline
Daniel Alves da Silva & 0.00 & 11 & 11.00 & Inf\\
\hline
Ronaldo de Assis Moreira & 6.53 & 11 & 4.47 & 0.69\\
\hline
\end{tabular}
\end{table}

The next section will present some shots visually that might help
explain these differences.

\hypertarget{visualising-shots}{%
\subsection{Visualising shots}\label{visualising-shots}}

Now we plot some of the shots to illustrate how such differences can
occur between reality and prediction, and also to show some chances that
had a high Scoring Probability, but were missed, or had a low
probability, but still went in. We used the package
\textbf{soccermatics} (Gallagher 2018) to plot the empty pitch, and then
we wrote our own function that places the players and the shot itself on
the pitch.

Figures \ref{fig:alves1} and \ref{fig:alves2} show two goals from Daniel
Alves that he scored against the odds. These efforts contribute to him
performing much better, than expected (see the table in the previous
section).

\begin{figure}[!h]

{\centering \includegraphics[width=0.65\linewidth]{Written-project_files/figure-latex/unnamed-chunk-9-1} 

}

\caption{\label{fig:alves1} Daniel Alves goal (Scoring Probability =  0.029)}\label{fig:unnamed-chunk-9}
\end{figure}

\begin{figure}[!h]

{\centering \includegraphics[width=0.65\linewidth]{Written-project_files/figure-latex/unnamed-chunk-10-1} 

}

\caption{\label{fig:alves2} Daniel Alves goal (Scoring Probability =  0.038)}\label{fig:unnamed-chunk-10}
\end{figure}

\newpage

Other shots however, were not converted despite having a large Scoring
Probability. Ludovic Giuly's header for example flew over the crossbar
(Figure \ref{fig:giuly}).

\begin{figure}[!h]

{\centering \includegraphics[width=0.65\linewidth]{Written-project_files/figure-latex/unnamed-chunk-11-1} 

}

\caption{\label{fig:giuly} Ludovic Giuly miss (Scoring Probability =  0.886)}\label{fig:unnamed-chunk-11}
\end{figure}

Samuel Eto'o failed to score from a relatively easy position (Figure
\ref{fig:etoo}) against Real Madrid.

\begin{figure}[!h]

{\centering \includegraphics[width=0.65\linewidth]{Written-project_files/figure-latex/unnamed-chunk-12-1} 

}

\caption{\label{fig:etoo} Samuel Eto'o miss (Scoring Probability =  0.823)}\label{fig:unnamed-chunk-12}
\end{figure}

\newpage

Finally a typical Lionel Messi goal against Levante, where he had less
than 50\% probability of scoring, but still managed to put the ball into
the net.

\begin{figure}[!h]

{\centering \includegraphics[width=0.65\linewidth]{Written-project_files/figure-latex/unnamed-chunk-13-1} 

}

\caption{\label{fig:messi} Lionel Messi goal (Scoring Probability =  0.120)}\label{fig:unnamed-chunk-13}
\end{figure}

\newpage

\hypertarget{future-improvements}{%
\section{Future improvements}\label{future-improvements}}

The limitations of this study must be acknowledged. The data sample
consist only of \textbf{F.C. Barcelona} games in the national regular
season \textbf{La Liga}. From this regards, Barcelona players - which
are known to be very effective - playing against worse teams can lead to
some biased results. It is also worth to be mentioned than as the
dataset starts in 2005, many of the shots are done by \textbf{Leo
Messi}, for many the best player in history and definetly one of the
best strikers of all times. This can also lead to not very relevant
results for the football comunity. This is why this project used player
ratings, as an attempt to introduce this information to the models, that
can be used therefore at all levels. It would be good therefore, to add
more shot situations from many different competitions from different
countries and different teams. This would give a more realistic dataset
and also it would provide more data points that would definetly help
train the algorithm, probably achieving better results.

Another possible improvement would be to create more variables. With the
dataset provided the authors created a data set that comprised all those
variables that could have an effect on the result of the shot. New
variables were computed giving a proper dataset to work on. However,
more variables could have been added or created. With more variables, a
better picture of the situation of the shot is drawn. With this new
dataset other Machine Learning approaches, more modern and complex could
have been applied. For instance, would be good to train a Neural Net in
order to predict goals. In order to do so, high computational resources
are needed, that is why Neural Nets are not used in this studio, but
could probably improve the results.

Another improvement to the study would be to apply the same methods to
other types of events other than \textbf{shots}, such as success in
\textbf{passes} or \textbf{tackles}. This would provide information not
only to the strikers but to every player on the field, improving - as
this is the final aim of Football Analysis - the game.

\hypertarget{conclusion}{%
\section{Conclusion}\label{conclusion}}

Football is a very demanding sport, and the high impact that it has in
society makes teams strive for perfection. From this point, game demands
are extensively analysed and new concepts coming from data science are
introduced to coaches plans. This projects created a classifier that
provides more information regarding the shooting positions of players.
It intends to help players choose wiser when shooting, or in the other
hand to advise them not to shoot when the chances are very low and a
pass could create a better situation. Also it provides information, as
in this project not only the situation but the players involved are
studied, about which palyers should take which shots. It is commonly
known that best players take more shots, but there are situations where
a player which apparently is not that good is the best fit for that
particular shot in that particular situation.

The algorithm, built on a simple logistic regression and only 25
relevant features classified the Scoring Probability with a combined
precision and recall of 35.24\% (F1-score), a recall of 23.72\% and an
accuracy of 85.89\% improving the results of the xG study from
statsBomb. This study tries to improve the football analysis comunity
and football itself providing a tool to assess shooting scenarios in
professional football.

\hypertarget{references}{%
\section*{References}\label{references}}
\addcontentsline{toc}{section}{References}

\hypertarget{refs}{}
\begin{cslreferences}
\leavevmode\hypertarget{ref-StAc19}{}%
Academy, StatsBomb. n.d. ``StatsBomb Open Dataset.''
\url{https://statsbomb.com/academy/}.

\leavevmode\hypertarget{ref-FIFA}{}%
Arts, Electronic. n.d. ``FIFA.''
\url{https://www.ea.com/en-gb/games/fifa/fifa-21}.

\leavevmode\hypertarget{ref-random}{}%
Breiman, Cutler, Leo, and Matthew Wiener. 2018. ``Breiman and Cutler's
Random Forests for Classification and Regression.'' \emph{CRAN}.

\leavevmode\hypertarget{ref-omni}{}%
Calculator, Omni. n.d. ``How to Find the Angle of a Triangle.''
\url{https://www.omnicalculator.com/math/triangle-angle\#how-to-find-the-angle-of-a-triangle}.

\leavevmode\hypertarget{ref-Findex}{}%
FIFAindex. n.d. ``Player Stats Database.''
\url{https://www.fifaindex.com/players/top/}.

\leavevmode\hypertarget{ref-soccermatics}{}%
Gallagher, Joe. 2018. ``Visualise Spatial Data from Soccer Matches.''
\emph{CRAN}.

\leavevmode\hypertarget{ref-geek}{}%
Geeks, Geeks for. n.d. ``Check Whether a Given Point Lies Inside a
Triangle or Not.''
\url{https://www.geeksforgeeks.org/check-whether-a-given-point-lies-inside-a-triangle-or-not/}.

\leavevmode\hypertarget{ref-boost}{}%
Hofner, Benjamin. 2020. ``Model-Based Boosting.'' \emph{CRAN}.

\leavevmode\hypertarget{ref-BeLa19}{}%
Larrousse, Benjamin. 2019. ``Improving Decision Making for Shots.''
\emph{StatsBomb}.

\leavevmode\hypertarget{ref-glm}{}%
Schlegel, Benjamin. 2019. ``Predicted Values and Discrete Changes for
Glm.'' \emph{CRAN}.
\end{cslreferences}


\end{document}
